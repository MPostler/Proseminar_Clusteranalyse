\chapter{Eigenschaften von Clusterlösungen}

\section{Klassifizierung von Objekten und/oder Merkmalen}
Nach Bankhofer \cite{Bankhofer.2008} können erfasste Merkmale in quantitative und qualitative Merkmale unterschieden werden.
Quantitative Merkmal besitzen einen hohen Informationsgehalt und ihre Ausprägungen werden mit Zahlen benannt. Qualitative Merkmale besitzen einen niedriegeren Informationsgehalt als quantitative Merkmale und werden durch Begriffe beschrieben. Qualitative Merkmale können weiterhin in nominale und in ordinale Merkmale unterschieden werden.

\begin{enumerate}
        \item Quanitative Merkmale: Diese Merkmale werden auch kardinale oder metrische Merkmale genannt. Sie besitzen den höchstmöglichen Informationsgehalt. Die Merkmalsausprägungen sind Zahlen, welche eine Ordnung besitzen. Dabei können alle möglichen Merkmalsausprägungen auf Skalen geordnet werden. Dadurch kann man den Abstand, sowie das Verhältnis zwischen zei Merkmalsausprägungen bestimmten.
        \item Ordinale Merkmale: Diese Merkmale gehören zu den qualitativen Merkmalen. Die Merkmalsausprägungen werden durch Begriffe dargestellt. Dabei können alle Ausprägungen vollständig geordnet werden. Durch diese Ordnung ist ein Vergleich zweier Merkmalsausprägungen möglich, wodurch man die Merkmalsausprägungen in Reihenfolgen bringen kann.
        \item Nominale Merkmale: Diese Merkmale gehören zu den qualitativen Merkmalen. Die Merkmalsausprägungen werden durch Begriffe dargestellt. Die Ausprägungen besitzen dabei keine Ordnung. Die Ausprägungen können lediglich auf Gleichheit oder Ungleichheit überprüft werden. Diese Merkmale besitzen den niedrigsten Informationsgehalt.
\end{enumerate}

\section{Disjunkte und nicht-disjunkte Zuordnung}
Nach Bankhofer \cite{Bankhofer.2008} können Clusterlösungen in disjunkte und nicht disjunkte Lösungen unterteilt werden. Diese Klassizifierung der Cluster beschreibt ob Objekte mehreren Klassen zugeordnet werden können oder nur Bestandteil einer Klasse sind.
\begin{enumerate}
    \item Disjunkte Lösungen: Ein ist Objekt genau einer Klasse zugeordnet. Es kommtg also nicht zu Überschneidungen von mehreren Klassen.
    \item Nicht-Disjunkte Lösungen: Ein Objekt kann mehreren Klassen zugeordnet werden. Es können also Klassen existieren, welche gemeinsame Objekte besitzen. Diese Klassen können auch als überlappende oder überdeckende Klassen bezeichnet werden. Dabei ist zu beachten, dass Teilmengenbeziehungen zwischen Klassen ausgeschlossen sind.
\end{enumerate}

\section{Exhaustive und nicht-exhaustive Zuordnung}
Nach Bankhofer \cite{Bankhofer.2008} können Cluster unterteilt werden, ob alle Objekte der Datenmenge bei der Klassifizierung einbezogen werden oder nicht. Aufgrundlage dieses Klassifikationstyps kann man Cluster in 2 Gruppen unterteilen:
\begin{enumerate}
    \item Exhaustive Zuordnung: Alle Objekte der verarbeiteten Datenmenge werden klassifiziert.
    \item Nicht Exhaustive Zuordnung: Nur ein Teil der verarbeiteten Datenmenge wird klassifiziert. Die nicht berücksichtigten Objekte werden vernachlässigt.
\end{enumerate}

\section{Homogenität innerhalb der Cluster}
Die Homogenität innerhalb eines Clusters beschreibt wie ähnlich sich die Objekte des Clusters sind. Die Objekte innerhalb eines Clusters sollten maximal homogen, also maximal ähnlich zu einander sein (vgl. Bacher et al \cite{Bacher.2010}, S.16).
Nach Bankhofer \cite{Bankhofer.2008} kann die Homogenität durch die Innerklassenverschiedenheit abgebildet werden. Dies wird anhand der Maximaldistanz zwischen Objekten des Clusters dargestellt. Bei einelementigen Clustern ist dabei die Homogenität maximal.

Formeln hinzufügen ???
\section{Heterogenität zwischen den Clustern}
Die Heterogenität zwischen den Clustern beschreibt, wie verschieden die Objekte von verschiedenen Clustern sind. Die Objekte von verschiedenen Clustern sollen minimal homogen sein. Die Objekte sollen heterogen, also verschieden sein (vgl. Bacher et al \cite{Bacher.2010}, S.16).
Nach Bankhofer \cite{Bankhofer.2008} kann die Heterogenität durch die Zwischenklassenverschiedenheit beschrieben werden. Diese Verschiedenheit wird anhand von Distanzen dargestellt. Diese Distanzen können auf Grundlage von verschiedenen Methoden berechnet werden:
\begin{enumerate}
        \item Single Linkage: Die minimale Distanz zwischen zwei Objekten der betrachteten Cluster wird zur Darstellung genutzt.
        \item Average Linkage: Die mittlere Distanz zwischen den Objekten der betrachteten Cluster wird zur Darstellung genutzt.
        \item Complete Linkage: Die maximale Distanz zwischen zwei Objekten der betrachteten Cluster wird zur Darstellung genutzt.
\end{enumerate}
