\chapter{Eigenschaften von Clusterlösungen}

\section{Klassifizierung von Objekten und/oder Merkmalen}
\section{Disjunkte und nicht-disjunkte Zuordnung}
Nach \cite{Bankhofer.2008} können Clusterlösungen in disjunkte und nicht disjunkte Lösungen unterteilt werden. Diese Klassizifierung der Cluster beschreibt ob Objekte mehreren Klassen zugeordnet werden können oder nur Bestandteil einer Klasse sind. 
Bei disjunkten Lösungen ist ein Objekt genau einer Klasse zugeordnet. Es kommtg also nicht zu Überschneidungen von mehreren Klassen.
Bei nicht disjunkten Lösungen kann ein Objekt mehreren Klassen zugeordnet werden. Es können also Klassen existieren, welche gemeinsame Objekte besitzen. Diese Klassen können auch als überlappende oder überdeckende Klassen bezeichnet werden. Dabei ist zu beachten, dass Teilmengenbeziehungen zwischen Klassen ausgeschlossen sind.
\section{Exhaustive und nicht-exhaustive Zuordnung}
\section{Homogenität innerhalb der Cluster}
\section{Heterogenität zwischen den Clustern}
