\chapter{Eigenschaften von Clusterlösungen}

\section{Klassifizierung von Objekten und/oder Merkmalen}
Nach Bankhofer/Vogel \cite{Bankhofer.2008}, S.158-163 können erfasste Merkmale in quantitative und qualitative Merkmale unterschieden werden.
Quantitative Merkmal besitzen einen hohen Informationsgehalt und ihre Ausprägungen werden mit Zahlen benannt. Qualitative Merkmale besitzen einen niedrigeren Informationsgehalt als quantitative Merkmale und werden durch Begriffe beschrieben. Qualitative Merkmale können weiterhin in nominale und in ordinale Merkmale unterschieden werden.

\begin{enumerate}
        \item \textit{Quantitative Merkmale}: Diese Merkmale werden auch kardinale oder metrische Merkmale genannt. Sie besitzen den höchstmöglichen Informationsgehalt. Die Merkmalsausprägungen sind Zahlen, welche eine Ordnung besitzen. Dabei können alle möglichen Merkmalsausprägungen auf Skalen geordnet werden. Dadurch kann man den Abstand, sowie das Verhältnis zwischen zwei Merkmalsausprägungen bestimmten.
        \item \textit{Ordinale Merkmale}: Diese Merkmale gehören zu den qualitativen Merkmalen, wobei alle Merkmalsausprägungen durch Begriffe dargestellt werden und vollständig geordnet werden können. Durch diese Ordnung ist ein Vergleich zweier Merkmalsausprägungen möglich, wodurch man die Merkmalsausprägungen in Reihenfolgen bringen kann.
        \item \textit{Nominale Merkmale}: Diese Merkmale gehören ebenfalls zu den qualitativen Merkmalen. Die Merkmalsausprägungen werden durch Begriffe dargestellt, wobei die Ausprägungen keine Ordnung besitzen. Die Ausprägungen können somit lediglich auf Gleichheit oder Ungleichheit überprüft werden. Diese Merkmale besitzen den niedrigsten Informationsgehalt.
\end{enumerate}

\section{Explorative und konfirmatorische Lösungen}
Clusterverfahren können nach Bacher et al. \cite{Bacher.2010}, S.22 in explorative und konfirmatorische Clusterverfahren unterschieden werden. 
Bei explorativen Verfahren steht die Anzahl der Cluster sowie die kennzeichnende Merkmalsausprägung nicht schon im Vorhinein fest.
Bei konfirmatorischen Verfahren steht die Anzahl der Cluster und die Charakteristik der Cluster bereits vor dem Anwenden der Verfahren fest und muss somit schon im Vorfeld des eigentlichen Clusteranalyseverfahrens ermittelt werden.

\section{Disjunkte und nicht-disjunkte Zuordnung}
Nach Bankhofer/Vogel \cite{Bankhofer.2008}, S.175 können Clusterlösungen in disjunkte und nicht disjunkte Lösungen unterteilt werden. Diese Klassifizierung der Cluster beschreibt ob Objekte mehreren Klassen zugeordnet werden können oder nur Bestandteil einer Klasse sind.
\begin{enumerate}
    \item \textit{Disjunkte Lösungen}: Ein Objekt ist genau einer Klasse zugeordnet. Es kommt also nicht zu Überschneidungen von mehreren Klassen.
    \item {Nicht-Disjunkte Lösungen}: Ein Objekt kann mehreren Klassen zugeordnet werden. Es können also Klassen existieren, welche gemeinsame Objekte besitzen. Diese Klassen können auch als überlappende oder überdeckende Klassen bezeichnet werden. Dabei ist zu beachten, dass Teilmengenbeziehungen zwischen Klassen ausgeschlossen sind.
\end{enumerate}

\section{Exhaustive und nicht-exhaustive Zuordnung}
Nach Bankhofer/Vogel \cite{Bankhofer.2008}, S.174 können Cluster unterteilt werden, ob alle Objekte der Datenmenge bei der Klassifizierung einbezogen werden oder nicht. Auf Grundlage dieses Klassifikationstyps kann man Cluster in zwei Gruppen unterteilen:
\begin{enumerate}
    \item \textit{Exhaustive Zuordnung}: Alle Objekte der verarbeiteten Datenmenge werden klassifiziert.
    \item \textit{Nicht Exhaustive Zuordnung}: Nur ein Teil der Ausgangsdatenmenge wird klassifiziert. Die nicht berücksichtigten Objekte werden vernachlässigt.
\end{enumerate}

\section{Homogenität innerhalb der Cluster}
Die Homogenität innerhalb eines Clusters beschreibt wie ähnlich sich die Objekte des Clusters sind. Die Objekte innerhalb eines Clusters sollten maximal homogen, also maximal ähnlich zu einander sein (Vgl. Bacher et al \cite{Bacher.2010}, S.16).
Nach Bankhofer/Vogel \cite{Bankhofer.2008}, S.181 kann die Homogenität durch die Innerklassenverschiedenheit abgebildet werden. Dies wird anhand der Maximaldistanz zwischen Objekten des Clusters dargestellt. Bei einelementigen Clustern ist dabei die Homogenität maximal.

\colorbox{red}{Formeln hinzufügen ???}
\section{Heterogenität zwischen den Clustern}
Die Heterogenität zwischen den Clustern beschreibt, wie verschieden die Objekte von verschiedenen Clustern sind. Die Objekte von verschiedenen Clustern sollen minimal homogen sein. Die Objekte sollen heterogen, also verschieden sein (vgl. Bacher et al \cite{Bacher.2010}, S.16).
Nach Bankhofer/Vogel \cite{Bankhofer.2008}, S.181 kann die Heterogenität durch die Zwischenklassenverschiedenheit beschrieben werden. Diese Verschiedenheit wird anhand von Distanzen dargestellt. Diese Distanzen können auf Grundlage von verschiedenen Methoden berechnet werden:
\begin{enumerate}
        \item Single Linkage: Die minimale Distanz zwischen zwei Objekten der betrachteten Cluster wird zur Darstellung genutzt.
        \item Average Linkage: Die mittlere Distanz zwischen den Objekten der betrachteten Cluster wird zur Darstellung genutzt.
        \item Complete Linkage: Die maximale Distanz zwischen zwei Objekten der betrachteten Cluster wird zur Darstellung genutzt.
\end{enumerate}
\colorbox{red}{Verwechslung mit Verfahren? eventuell hier genauer abgrenzen}

\section{Fusionierungseigenschaften}
Clusterverfahren lassen sich nach Backhaus et al. (\cite{Backhaus.2016} S.488/489) anhand ihrer Fusionierungseigenschaften in drei Gruppen unterteilen:
\begin{enumerate}
        \item \textit{Dilatierende Verfahren}: Bei diesen Verfahren werden die Objekte in einzelne etwa gleich große Gruppen zusammengefasst.
        \item \textit{Kontrahierende Verfahren}: Bei diesen Verfahren stehen viele kleine Gruppen wenigen großen Gruppen gegenüber.
        \item \textit{Konservative Verfahren}: Diese Verfahren weisen weder kontrahierende noch dilatierende Merkmale auf.
\end{enumerate}
