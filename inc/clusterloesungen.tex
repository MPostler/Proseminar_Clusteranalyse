\chapter{Eigenschaften von Clusterlösungen}

\section{Klassifizierung von Objekten und/oder Merkmalen}
\section{Disjunkte und nicht-disjunkte Zuordnung}
Nach \cite{Bankhofer.2008} können Clusterlösungen in disjunkte und nicht disjunkte Lösungen unterteilt werden. Diese Klassizifierung der Cluster beschreibt ob Objekte mehreren Klassen zugeordnet werden können oder nur Bestandteil einer Klasse sind.
\begin{enumerate}
    \item Disjunkte Lösungen: Ein ist Objekt genau einer Klasse zugeordnet. Es kommtg also nicht zu Überschneidungen von mehreren Klassen.
    \item Nicht-Disjunkte Lösungen: Ein Objekt kann mehreren Klassen zugeordnet werden. Es können also Klassen existieren, welche gemeinsame Objekte besitzen. Diese Klassen können auch als überlappende oder überdeckende Klassen bezeichnet werden. Dabei ist zu beachten, dass Teilmengenbeziehungen zwischen Klassen ausgeschlossen sind.
\end{enumerate}
\section{Exhaustive und nicht-exhaustive Zuordnung}
Nach \cite{Bankhofer.2008} können Cluster unterteilt werden, ob alle Objekte der Datenmenge bei der Klassifizierung einbezogen werden oder nicht. Aufgrundlage dieses Klassifikationstyps kann man Cluster in 2 Gruppen unterteilen:
\begin{enumerate}
    \item Exhaustive Zuordnung: Alle Objekte der verarbeiteten Datenmenge werden klassifiziert.
    \item Nicht Exhaustive Zuordnung: Nur ein Teil der verarbeiteten Datenmenge wird klassifiziert. Die nicht berücksichtigten Objekte werden vernachlässigt.
\section{Homogenität innerhalb der Cluster}
\section{Heterogenität zwischen den Clustern}
