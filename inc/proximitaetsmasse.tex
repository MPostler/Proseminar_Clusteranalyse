\chapter{Proximitätsmaße}

Um die Einteilung in möglichst homogene Gruppen vornehmen zu können, müssen die zu untersuchenden Objekte bezüglich ihrer zu beobachtenden Eigenschaften auf Ähnlichkeit untersucht werden. Die Wahl eines Proximitätsmaß für eine Clusteranalyse hängt maßgeblich davon ab, ob die Clusteranalyse Objekte oder Variablen zu klassifizieren versucht (Vgl. Bacher et al. \cite{Bacher.2010}, S. 196).

Nach Backhaus et al \cite{Backhaus.2016} lassen sich zwei Arten von Proximitätsmaßen unterscheiden:

\begin{enumerate}
	\item \textit{Ähnlichkeitsmaße}: Diese Maße spiegeln die Ähnlichkeit zweier Objekte wider. Je höher der zugewiesene Wert für zwei Objekte, desto höher ist auch ihre Ähnlichkeit.
	\item \textit{Distanzmaße} (auch \textit{Unähnlichkeitsmaße}): Diese drücken die Unähnlichkeit zweier Objekte aus. Je größer die angegebene Distanz, desto unähnlicher sind die Objekte, wobei eine Distanz von Null ausdrückt, dass die zwei Objekte hinsichtlich ihrer Klassifikationsmerkmale vollkommen identisch sind. Die Distanzmaße lassen sich als entgegengesetzter Pol der Ähnlichkeitsmaße auffassen, wobei diese Eigenschaft die Überführung beider Maße ineinander ermöglicht. (Eckey et al. \cite{Eckey.2002}, S. 205).
\end{enumerate}

\cite{Bacher.2010} S. 200 Alle Ähnlichkeitsmaße \textit{ä} lassen sich durch 
\begin{equation}
u_{ij} = 1 - \ddot{a}_{ij} \quad \text{bzw.} \quad u_{g,g*} = 1- \ddot{a}_{g,g*}
\end{equation}
in Distanzmaße \textit{u} umwandeln.

Durch unterschiedliche Skalenniveaus der betrachteten Merkmale lassen sich eine Vielzahl von unterschiedlichen Proximitätsmaßen bestimmen.

\section{Proximitätsmaße für dichotome Merkmale}
Es kann von dichotomen Merkmalen (auch binären Merkmalen) gesprochen werden, wenn deren Modalitäten mit 0 und 1 kodiert werden; polytome Merkmale hingegen besitzen mehrere Ausprägungen. (Vgl. Eckey et al. \cite{Eckey.2002}, S. 218). 

Um die Kombinationen der Merkmalsausprägungen bei einem Objekt- bzw. Variablenpaar zu definieren, kommt oft eine Kontingenztabelle (auch \textit{Vierfeldertafel}) zum Einsatz: \\

\begin{tabular}{cc|cc|c}
	& & \multicolumn{2}{c|}{Variable \textit{j} oder Objekt \textit{g*}} & \\ 
	& & 0 & 1 & $\sum$ \\ \hline
	\multirow{2}{*}{Variable \textit{i} oder Objekt \textit{g}} & 0 (Nichtbesitz) & a & b & a + b \\
	& 1 (Besitz) & c & d & c + d \\ \hline
	& $\sum$ & a + c & b + d & a + b + c + d = m \\ 
\end{tabular}
\bigskip
\\
Die Variablen \textit{a} und \textit{d} geben Auskunft über Übereinstimmungen. Die Variable \textit{a} gibt dabei Auskunft zur Übereinstimmung des Nichtbesitzes, dass bedeutet beide betrachteten Objekte besitzen das untersuchte Merkmal nicht. Die Variable \textit{d} gibt Auskunft zur Übereinstimmung des Besitzes, dass bedeutet beide betrachteten Objekte besitzen das untersuchte Merkmal. Die Variablen \textit{b} und \textit{c} geben Auskunft zur Nichtübereinstimmungen, dass heißt nur eines der Objekte besitzt das untersuchte Merkmal.

\colorbox{red}{überarbeiten}Ausgehend von \cite{Bacher.2010}, S. 199
\begin{equation}
\ddot{a}_{ij} \quad \text{bzw.} \quad \ddot{a}_{g,g*} = \frac{\alpha \cdot a + \beta \cdot d}{\delta \cdot a + \beta \cdot d + \gamma \cdot (b + c)}
\end{equation}

lassen sich durch die unterschiedlichen Gewichtungsfaktoren $\alpha$, $\beta$, $\gamma$ und $\delta$ verschiedene Ähnlichkeitsmaße herleiten. Dabei gehen je nach Ähnlichkeitsmaß manche Variablen mehr und manche weniger in die Berechnung ein. Die Ähnlichkeitsmaße sind nach dem jeweiligen Entwickler benannt.
Nachfolgend sind einige Ähnlichkeitsmaße, nach Bacher \cite{Bacher.2010}, S. 200, mitsamt ihrer Formel dargestellt. \\ 
\\
%\begin{tabular}{|l*{1}{|m{3.5cm}}|c*{1}{|m{3.5cm}}|p{10cm}|}
\begin{tabular}{|l|c|p{8cm}|}
	\hline
	\rowcolor{babyblueeyes}Ähnlichkeitsmaß & Berechnungsformel & Eigenschaften \\ \hline
	\rowcolor{beaublue}Jaccard I & $\frac{0 \cdot \textit{a} + 1 \cdot \textit{d}}{1 \cdot \textit{a} + 1 \cdot \textit{d} + 1 \cdot (\textit{b} + \textit{c})} $ & Gemeinsamer Nichtbesitz geht nicht in Berechnung ein \\ \hline
	\rowcolor{beaublue}Dice & $ \frac{0 \cdot \textit{a} + 2 \cdot \textit{d}}{1 \cdot \textit{a} + 1 \cdot \textit{d} + 1 \cdot (\textit{b} + \textit{c})} $ & Gemeinsamer Nichtbesitz geht nicht in Berechnung ein, gemeinsamer Besitz doppelt \\ \hline
	\rowcolor{beaublue}Sokal \& Sneath I & $ \frac{0 \cdot \textit{a} + 1 \cdot \textit{d}}{1 \cdot \textit{a} + 1 \cdot \textit{d} + 2 \cdot (\textit{b} + \textit{c})} $ & Gemeinsamer Nichtbesitz geht nicht in Berechnung ein, Nichtübereinstimmung doppelt \\ \hline
	\rowcolor{beaublue}Russel \& Rao & $ \frac{0 \cdot \textit{a} + 1 \cdot \textit{d}}{1 \cdot \textit{a} + 1 \cdot \textit{d} + 1 \cdot (\textit{b} + \textit{c})} $ & Gemeinsamer Nichtbesitz wird nicht als Ähnlichkeit betrachtet, geht aber in den Nenner ein \\ \hline
	\rowcolor{beaublue}Simple-Matching & $ \frac{1 \cdot \textit{a} + 1 \cdot \textit{d}}{1 \cdot \textit{a} + 1 \cdot \textit{d} + 1 \cdot (\textit{b} + \textit{c})} $ & Alles wird gleich gewichtet \\ \hline
	\rowcolor{beaublue}Sokal \& Sneath II & $ \frac{2 \cdot (\textit{a} + \textit{d})}{2 \cdot (\textit{a} + \textit{d}) + 1 \cdot (\textit{b} + \textit{c})} $ & Übereinstimmungen werden doppelt gewichtet \\ \hline
	\rowcolor{beaublue}Rogers \& Tanimoto & $ \frac{1 \cdot \textit{a} + 1 \cdot \textit{d}}{1 \cdot a + 1 \cdot \textit{d} + 2 \cdot (\textit{b} + \textit{c})} $ & Nichtübereinstimmung wird doppelt gewichtet \\ \hline
\end{tabular}
	

\section{Proximitätsmaße für polytome Merkmale}
Polytome Merkmale können auf einem nominalen oder ordinalen Skalenniveau gemessen werden und immer anhand einer Dichotomisierung durch mehrere dichotome Merkmale (sogenannte Dummy-Variablen) ersetzt werden. Ein Merkmal mit \textit{r} Merkmalsausprägungen erfordert somit \textit{r} Dummy-Variablen (Vgl. Bankhofer/Vogel \cite{Bankhofer.2008}, S. 159). Hierbei ist aber vor allem darauf zu achten, dass ein erheblicher Informationsverlust bei der Dichotomisierung ordinal skalierter Merkmale in Kauf genommen werden muss. In diesem Fall würde sich eher eine Distanzmessung über die Rangdistanz anbieten, die vorteilhaft anwendbar ist, sofern die Ränge als Intervallskala interpretiert werden und die meisten Merkmale sich in ihren Ausprägungen unterscheiden (Vgl. Eckey et al. \cite{Eckey.2002}, S. 225).

\section{Proximitätsmaße für hierarchische Merkmale}

Bei hierarchischen Merkmalen wird von den einzelnen Hierarchieebenen ausgegangen. Ihnen wird ein Wert zugewiesen und ausgehend vom Zusammentreffen der untergeordneten Merkmalsausprägungen auf einer Hierarchieebene wird ihnen die Distanz der Hierarchieebene zugewiesen (Vgl. Bankhofer/Vogel \cite{Bankhofer.2008}, S. 160).

\section{Proximitätsmaße für quantitative Merkmale}


