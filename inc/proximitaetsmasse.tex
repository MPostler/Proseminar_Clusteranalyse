\chapter{Proximitätsmaße}

Um die Einteilung in möglichst homogene Gruppen vornehmen zu können, müssen die zu untersuchenden Objekte bezüglich ihrer zu beobachtenden Eigenschaften auf Ähnlichkeit untersucht werden. 

Nach Backhaus et al \cite{Backhaus.2016} lassen sich zwei Arten von Proximitätsmaßen unterscheiden:

\begin{enumerate}
	\item Ähnlichkeitsmaße: Diese Maße spiegeln die Ähnlichkeit zweier Objekte wider. Je höher der zugewiesene Wert für zwei Objekte, desto höher ist auch ihre Ähnlichkeit.
	\item Distanzmaße: Diese drücken die Unähnlichkeit zweier Objekte aus. Je größer die angegebene Distanz, desto unähnlicher sind die Objekte, wobei eine Distanz von Null ausdrückt, dass die zwei Objekte hinsichtlich ihrer Klassifikationsmerkmale vollkommen identisch sind. Die Distanzmaße lassen sich als entgegengesetzter Pol der Ähnlichkeitsmaße auffassen, wobei diese Eigenschaft die Überführung beider Maße ineinander ermöglicht. (Eckey et al \cite{Eckey.2002}). 
\end{enumerate}

\cite{Eckey.2002} wenn mathematische Eigenschaften noch mit aufgenommen werden sollen

Durch unterschiedliche Skalenniveaus der betrachteten Merkmale lassen sich eine Vielzahl von unterschiedlichen Proximitätsmaßen bestimmen.