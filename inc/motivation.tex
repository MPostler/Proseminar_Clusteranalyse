\chapter{Motivation}

Für die Klassifizierung wurden im Laufe der Geschichte zahlreiche Verfahren entworfen, die sich in Bezug auf die Vorgehensweise und ihren Anforderungen an die Datenstruktur unterscheiden. Bereits Anfang des vergangenen Jahrhunderts begannen Biologen und Anthropologen, Klassifizierungen anzuwenden, wobei ihre Verfahren eher intuitiv als methodisch begründet waren. Die ersten methodischen Verfahren wurden in den späten 1930er bzw. 1940er Jahren durch die Sozialwissenschaften, insbesondere die Psychologie, entwickelt und etablierten sich erst in den 1960er Jahren. Inzwischen sind Clusteranalysen umfangreich untersucht und finden Anwendung in sehr vielen Bereichen, nicht zuletzt weil die auswertbaren Datenmengen skalieren. So werden Clusteranalysen heute beispielsweise in der Biologie zur Untersuchung von Genom-Sequenzen, im Marketing zur Identifizierung von Kundengruppen oder auch klassisch in der Städte-Statistik zur Erkennung und Differenzierung von Bevölkerungsgruppen genutzt. Zudem haben sich Klassifizierungsverfahren ebenfalls in der Politik zur Bestimmung förderungsrelevanter Regionen nützlich gemacht. \\

So bedeutend diese Disziplin des Data Mining in der Gesellschaft ist, genauso wichtig ist auch ihre richtige Anwendung. Ziel dieser wissenschaftlichen Arbeit soll es sein, einen Überblick über die verschiedenen Verfahren zu gewinnen und Anwendungsempfehlungen bezüglich der Vorgehensweise bei unterschiedlichen Datenstrukturen und Anforderungsszenarien an die Clusteranalyse zu geben. \\

Dazu gehen wir zunächst auf verschiedene Eigenschaften von Clusterlösungen ein, bevor wir uns der Errechnung der Proximitätsmaße als Grundlage für die Clusteranalyseverfahren zuwenden.
Anhand der genauen Untersuchung einzelner Verfahren, die sich in der Praxis bewährt haben, soll ein Regelwerk aufgestellt werden, das die Entscheidung für ein geeignetes Vorgehen anhand von genau definierten Entscheidungskriterien vereinfachen soll.