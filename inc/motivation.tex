\chapter{Motivation}

Eine Clusteranalyse oder auch Klassifizierung dient dazu, Objekte und/oder Merkmale zu klassifizieren. Dabei sollen Merkmale und/oder Objekte in möglichst homogenen Klassen, die untereinander möglichst heterogen sind, zusammengefasst werden.

Für die Klassifizierung wurden zahlreiche Verfahren entworfen, die sich in Bezug auf die Vorgehensweise und ihren Anforderungen an die Datenstruktur unterscheiden.

Ziel dieser wissenschaftlichen Arbeit soll es sein, einen Überblick über die verschiedenen Verfahren zu gewinnen und Anwendungsempfehlungen bezüglich der Vorgehensweise bei unterschiedlichen Datenstrukturen und Anforderungsszenarien an die Clusteranalyse zu geben. Anhand von Bewertungskriterien soll ein festes Regelwerk in Form eines Entscheidungsbaumes aufgestellt werden, das die Entscheidung für ein geeignetes Vorgehen ermöglichen soll.