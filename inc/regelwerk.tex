\chapter{Anwendungsempfehlungen}

\section{Allgemeine Anwendungsempfehlungen}

Nach \citet[S. 510]{Backhaus.2016} müssen einige Vorüberlegungen vorab angestellt werden, um das ideale Clusteranalyseverfahren auswählen zu können. Die Vorüberlegungen sind folgende:
\begin{enumerate}
    \item Anzahl der Objekte
    \item Problem der Ausreißer
    \item Anzahl der betrachteten Merkmale
    \item Gewichtung der Merkmale
    \item Vergleichbarkeit der Merkmale
\end{enumerate}

\citet[S. 275]{Bacher.2010} empfehlen das Weighted-Average-Linkage-Verfahren, da es Dilatations- und Kontraktionseffekte sowie Inversionen vermeidet. Weiterhin ermöglicht das zugehörige Verschmelzungsschema im Gegensatz zu manch anderen Verfahren eine klare Interpretation. Es sollte eventuell doch auf ein anderes Verfahren zurückgegriffen werden, wenn die Daten nur ordinale Informationen aufweisen und Invarianz gegenüber monotonen Transformationen gewünscht ist oder wenn vermutet wird, dass sich die Cluster überlappen können. Bei Überlappungen hat sich das Complete Linkage-Verfahren bewährt.

Das Single Linkage-Verfahren sollte gewählt werden, wenn die Homogenität innerhalb der Cluster wichtiger ist als die Heterogenität zwischen den Clustern und wenn eher wenige, dafür größere Cluster erzeugt werden sollen. Allerdings muss die Tendenz zur Kettenbildung beachtet werden.

Das Complete Linkage-Verfahren sollte gewählt werden, wenn die Heterogenität wichtiger als die Innerklassengleichheit ist und wenn eher mehrere, kleinere Cluster erzeugt werden sollen. Dieses Verfahren eignet sich nicht, um Ausreißer zu identifizieren.

Das Average Linkage-Verfahren sollte genutzt werden, wenn Homogenität und Heterogenität ungefähr gleich gewichtet werden sollen.

Wird angenommen, dass ein Cluster nicht durch ein Objekt sinnvoll dargestellt werden kann, sondern mithilfe seines Zentrums dargestellt werden kann, sollte man das Median-, Centroid- oder Ward-Verfahren nutzen. Dabei eignet sich das Ward-Verfahren besonders bei hierarchischen Ähnlichkeitsbeziehung und wenn die Anzahl der Objekte gering ist. Das Median- und Centroid-Verfahren eignet sich schlechter als das Ward-Verfahren, da es zu Inversionen kommen kann. (Vgl.\citet[S.295]{Bacher.2010})

\section{Umgang mit Ausreißern}

Ein besonderes Problem stellen Ausreißer in den Daten dar. Viele Verfahren reagieren sensibel auf solche Werte, weswegen auf ein geeignetes Verfahren geachtet werden muss oder die Ausreißer sollten vorher erkannt und eliminiert werden. \\
Das Single Linkage-Verfahren erweist sich gegenüber Ausreißern als sehr robust. Aufgrund seiner Eigenschaften wird es sogar benutzt, um Ausreißer zu identifizieren, um sie später eliminieren zu können. Beim Complete Linkage-Verfahren lassen sich nur schlecht Ausreißer erkennen; das Verfahren ist aber auch mit Ausreißern in den Daten durchführbar. Beim Ward-Verfahren müssen vor der Ausführung die Ausreißer eliminiert werden, um fachlich gute Ergebnisse zu erhalten.

\section{Bewertung mithilfe des Rand-Index}

Die Anwendung mehrerer Clusteranalyseverfahren auf Grundlage der selben Datenmatrix führt oft zu einer ähnlichen Lösung und erlaubt dahingehend die Beurteilung der Brauchbarkeit einzelner Verfahren. Kommen mehrere Verfahren zum gleichen Ergebnis, bestätigt sich die zugrundeliegende Clusterstruktur meist. Hier ist jedoch zu beachten, dass die am häufigsten vorkommende Lösung nicht immer die optimale Lösung für den Problemfall darstellt, sondern auch eine Einzellösung für den Anwender brauchbar sein kann. \\
Ein Maß, um die Ähnlichkeit von Clusterlösungen bewerten zu können, ist der Rand-Index. Seine Grundlage bildet eine Vierfeldertafel, die aufzeigt, welche Objekte bei zwei unterschiedlichen Verfahren in den selben oder unterschiedlichen Clustern liegen:

\begin{tabular}{c>{\raggedright\arraybackslash}p{4cm}|>{\centering\arraybackslash}p{3.5cm}>{\centering\arraybackslash}p{3.5cm}|c}
	& & \multicolumn{2}{c|}{Verfahren B} & \\ 
	& & Paare im selben Cluster & Paare in verschiedenen Clustern & $\sum$ \\ \hline
	\multirow{2}{*}{Verfahren A} & Paare im selben Cluster & a & b & a + b \\
	& Paare in verschiedenen Clustern & c & d & c + d \\ \hline
	& $\sum$ & a + c & b + d & $\binom{n}{2}$ \\ 
\end{tabular}
\bigskip

Sind diese Beobachtungen erfasst, kann der Rand-Index gebildet werden:

\begin{equation}
RAND = \frac{a+d}{\binom{n}{2}}
\end{equation}

Basis hierfür ist die gleiche Clusteranzahl in beiden Lösungen. Ist dies nicht gegeben, so muss der Rand-Index mit einer Korrekturgröße ergänzt werden:

 \begin{gather}
 RAND = \frac{a+d-f_c}{\binom{n}{2}-f_c} \quad \text{mit} \\
 f_c = \frac{n\cdot(n^2+1)-(n+1)\cdot\sum_{g=1}^{G}n_{g \bullet}^2-(n+1)\cdot\sum_{h=1}^{H}n_{\bullet h}^2+2\cdot\sum_{g=1}^{G}\sum_{h=1}^{H}n_{g \bullet}^2n_{ \bullet h}^2/n}{2\cdot(n-1)}
 \end{gather}
 
 $G$ und $H$ sind die beiden zuvor ermittelten Clusterzahlen der Lösungen und $n_{g \bullet}$ bzw. $n_{ \bullet h}$ ergeben sich aus der Kreuztabellierung der Objekte hinsichtlich der erfolgten Clusterzuordnung: \\
 
\begin{center}
\begin{tabular}{|c|c|c|c|c|c|} 
	\hline
	& \multicolumn{4}{c|}{Verfahren B} & \\ \hline
	Verfahren A & $C_1$ & $C_2$ & \dots & $C_H$ & $n_g$ \\ \hline
	$C_1$ & $n_{11}$ & $n_{12}$ & \dots & $n_{1H}$ & $n_{1 \bullet}$ \\
	\vdots & \vdots & \vdots & \vdots & \vdots & \vdots \\
	$C_G$ & $n_{G1}$ & $n_{G2}$ & \dots & $n_{GH}$ & $n_{4 \bullet}$ \\ \hline
	$n_{\bullet h}$ & $n_{\bullet 1}$ & $n_{\bullet 2}$ & \dots & $n_{\bullet H}$ & $n$ \\ \hline
\end{tabular}
\end{center}
\bigskip

Der Rand-Index liegt im Wertebereich [0;1] und nimmt mit dem Grad der Konkordanz der Clusterlösungen zu. Ein maximaler Wert von 1 steht dabei für die Identität der beiden Lösungen \citep[Vgl.][S. 270-272]{Everitt.2011}. 
