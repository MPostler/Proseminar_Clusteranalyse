\chapter{Hierarchisch-agglomerative Verfahren}

In diesem Kapitel sollen einzelne hierarchisch-agglomerative Verfahren untersucht werden, da sich Austausch- oder divisive Verfahren in der tatsächlichen Anwendung nicht bewährt haben. Der Grund hierfür ist die höhere Rechenzeit für Algorithmen der divisiven Verfahren in Verbindung mit dem Fehlen eines Nachweises, dass divisive Strategien präzisere Clusterlösungen liefern als agglomerative Verfahren. In der Praxis wird daher meist auf hierarchisch-agglomerative Verfahren zurückgegriffen (Vgl. Pedrycz \cite{Pedrycz.2010} S. ???)
Nach Piegorsch (\cite{Piegorsch.2015}, S. 378) kann der Austausch des zugrundeliegenden Verfahrens Fluch oder Segen sein: Die Verwendung eines anderen Verfahrens auf Basis der selben Daten zeigt interessante oder unerwartete Änderungen im Ergebnis der Clusteranalyse. Dies ist eine Beobachtung, die den zugrundeliegenden Prozess der Wissensschöpfung verbessern kann.

\section{Single Linkage-Verfahren}
Bei dem Single Linkage-Verfahren können sowohl Ähnlichkeits- als auch Distanzmaße verwendet werden. Dabei werden jeweils die zwei Cluster zusammengefasst, welche die kleinste Distanz zwischen sich aufweisen. Die Distanz entspricht dabei der kleinstmöglichen Distanz zwischen Objekten der verschiedenen Cluster. Nach dem Bilden eines neuen Clusters müssen die Distanzen jeweils neu ausgerechnet werden und die neue Minimaldistanz identifiziert werden.
Nach Backhaus \cite{Backhaus.2016} kann die neue Distanz vereinfacht mit folgender Formel errechnet werden:
\begin{equation}
D(R;P+Q) = min\{D(R,P);D(R,Q)\}
\end{equation}
Aufgrund des Vorgehens wird dieses Verfahren auch "Nearest-Neighbour-Verfahren" genannt. (Vgl. Eckey et al. \cite{Eckey.2002}, S. 231)
Dieses Verfahren ist ein kontrahierendes Verfahren. Zudem kann es genutzt werden um Ausreißer zu identifizieren. (Vgl. Backhaus \cite{Backhaus.2016}, S.481-483)
Dieses Verfahren erzeugt oft Cluster, die ziemlich diffus, langgestreckt und/oder unförmig sind. (Vgl. \cite{Piegorsch.2015}, S. 377)
Hierbei entsteht allerdings auch das Problem des Verkettungseffekts: Cluster werden zusammengefasst, die nur durch eine Brücke verbunden sind, im Raum aber deutlich separiert voneinander liegen. Dies kann zu eher heterogenen Clustern führen. (Vgl. Eckey et al. \cite{Eckey.2002}, S. 233)
Nach Clarke et al. (\cite{Clarke.2009}, S. 416) sind solche ungewöhnlichen Strukturen in der Natur allerdings durchaus üblich. Er zitiert: "...it's unclear in general wehether such properties are features or bugs." Der Einfluss solcher Fusionierungseigenschaften ist noch nicht vollständig untersucht.


\section{Complete Linkage-Verfahren}
Bei dem Complete Linkage-Verfahren können ebenfalls sowohl Ähnlichkeits- als auch Distanzmaße verwendet werden. Es werden jeweils die Cluster mit der geringsten Distanz zusammengefasst. Die Distanz berechnet sich allerdings anders als beim Single Linkage-Verfahren. Die Distanz zwischen Clustern entspricht hier nicht der kleinstmöglichen, sondern der größtmöglichen Distanz zwischen Objekten der verschiedenen Cluster. Auch hier muss nach jedem Zusammenfassen zweier Cluster die Distanzen jeweils neu ausgerechnet werden und die neue Maximaldistanz identifiziert werden.
Nach Backhaus \cite{Backhaus.2016} kann die neue Distanz vereinfacht mit folgender Formel errechnet werden:                                                                                                                                                                                                                                                                                                                                                                                  
\begin{equation}
D(R;P+Q) = max\{D(R,P);D(R,Q)\}
\end{equation}
Aufgrund des Vorgehens wird dieses Verfahren auch "Furthest-Neighbour-Verfahren" genannt.
Dieses Verfahren ist ein dilatierendes Verfahren. Es eignet sich im Gegensatz zum Single Linkage- Verfahren nicht gut, um Ausreißer zu identifizieren, da es eher kleine Gruppen bildet. (Vgl. Backhaus \cite{Backhaus.2016}, S.483/484) Ein Problem der Orientierung an den maximal entfernten Objekten zweier Cluster stellt das Ausbleiben einer Fusionierung dar, selbst wenn die mittlere Distanz dieser zweier Objekte keine merkliche Erhöhung der Heterogenität im neu zu bildenden Cluster anzeigen würde. (Vgl. Eckey et al. \cite{Eckey.2002}, S.236)

\section{Average Linkage-Verfahren}
\section{Centroid-Verfahren}
\section{Median-Verfahren}
\section{WARD-Verfahren}
Das Ward-Verfahren wird oft als sehr guter Clusteralgorithmus gesehen, da es im Vergleich zu anderen Verfahren sehr gute Partitionen findet und Objekte "richtig" den Clustern zuordnet. Dazu müssen jedoch folgende Bedingungen gegeben sein: Verwendung eines Distanzmaßes, alle Variablen metrisches Skalenniveau, keine Ausreißer vor Anwendung des Verfahrens, Variablen sind unkorelliert, Elementanzahl in jeder Gruppe ca. gleich groß )(zu erwarten),Cluster besitzen gleiche Ausdehnung. Dabei werden jeweils die Objekte zusammengefasst ,welche die Streuung innerhalb eines Cluster möglichst wenig erhöhen. Die Cluster werden also so gebildet, dass sie möglichst homogen sind. (Vgl. Backhaus \cite{Backhaus.2016} S.484) 

Ward neigt jedoch dazu, möglichst gleich große Cluster zu bilden und langgestreckte Cluster bzw. kleine Cluster mit wenigen Objekten nicht zu erkennen \cite{Backhaus.2016} S. 489, auch Grafik

+ Abbildung in \cite{Backhaus.2016} S.489??? Fusionierungseigenschaften -> Wie reagieren Verfahren?
Abbildung kann rein, Fusionierungseigenschaften vielleicht unter 2.Clusterlösungen erklären?


\begin{tabular}{|l|l|l|l|l|}
	\hline
	\rowcolor{babyblueeyes}Verfahren & Eigenschaft & Monoton? & Proximitätsmaße & Bemerkungen \\ \hline
	\rowcolor{beaublue}Single Linkage & kontrahierend & ja & alle & neigt zur Kettenbildung \\ \hline
	\rowcolor{beaublue}Complete Linkage & dilatierend & ja & alle & neigt zu kleinen Gruppen \\ \hline	
	\rowcolor{beaublue}Average Linkage & konservativ & ja & alle & \\ \hline
	\rowcolor{beaublue}Centroid & konservativ & nein & Distanzmaße & \\ \hline
	\rowcolor{beaublue}Median & konservativ & nein & Distanzmaße & \\ \hline
	\rowcolor{beaublue}WARD & konservativ & ja & Distanzmaße & bildet etwa gleich große Gruppen \\ \hline
\end{tabular}
