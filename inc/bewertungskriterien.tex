\chapter{Bewertungskriterien (erstmal leer lassen}

Für die Bewertung der verschiedenen Verfahren ist es notwendig, diese nach verschiedenen Kriterien zu untersuchen. Dabei stellt einer der größten Einflussfaktoren die externen Anforderungen an die Klassifizierung dar. Hier soll untersucht werden, was der Anwender mithilfe der Klassifizierung erreichen will und welche Eigenschaften verschiedene Klassifizierungen beinhalten können. \\
Einen weiteren Einflussfaktor stellt die zu untersuchende Merkmalsstruktur dar. Die ursprünglichen Merkmale können unterschiedliche Skalenniveaus beinhalten, wovon jede Skalierung eines Merkmals sich anders auf Verfahren der Klassifizierung auswirken kann. Problematisch sind hier vor allem auch Datenmatrizen, die Merkmale mit unterschiedlichen Skalenniveaus beinhalten. \\
Ein weiteres Kriterium, das betrachtet werden sollte, sind hier besondere Ausprägungen der zu untersuchenden Merkmale. Fehlende Werte, Ausreißer und irrelevante Variablen stellen eine potentielle Fehlerquelle bei der Klassifizierung dar.

\section{Fehlende Werten}
\section{Ausreißer}
\section{Irrelevante Variablen}

